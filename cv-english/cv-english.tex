%!TEX TS-program = xelatex
%!TEX encoding = UTF-8 Unicode
% Awesome CV LaTeX Template for CV/Resume
%
% This template has been downloaded from:
% https://github.com/posquit0/Awesome-CV
%
% Author:
% Claud D. Park <posquit0.bj@gmail.com>
% http://www.posquit0.com
%
%
% Adapted to be an Rmarkdown template by Mitchell O'Hara-Wild
% 23 November 2018
%
% Template license:
% CC BY-SA 4.0 (https://creativecommons.org/licenses/by-sa/4.0/)
%
%-------------------------------------------------------------------------------
% CONFIGURATIONS
%-------------------------------------------------------------------------------
% A4 paper size by default, use 'letterpaper' for US letter
\documentclass[11pt, a4paper]{awesome-cv}

% Configure page margins with geometry
\geometry{left=1.4cm, top=.8cm, right=1.4cm, bottom=1.8cm, footskip=.5cm}

% Specify the location of the included fonts
\fontdir[fonts/]

% Color for highlights
% Awesome Colors: awesome-emerald, awesome-skyblue, awesome-red, awesome-pink, awesome-orange
%                 awesome-nephritis, awesome-concrete, awesome-darknight

\definecolor{awesome}{HTML}{2f3640}

% Colors for text
% Uncomment if you would like to specify your own color
% \definecolor{darktext}{HTML}{414141}
% \definecolor{text}{HTML}{333333}
% \definecolor{graytext}{HTML}{5D5D5D}
% \definecolor{lighttext}{HTML}{999999}

% Set false if you don't want to highlight section with awesome color
\setbool{acvSectionColorHighlight}{true}

% If you would like to change the social information separator from a pipe (|) to something else
\renewcommand{\acvHeaderSocialSep}{\quad\textbar\quad}

\def\endfirstpage{\newpage}

%-------------------------------------------------------------------------------
%	PERSONAL INFORMATION
%	Comment any of the lines below if they are not required
%-------------------------------------------------------------------------------
% Available options: circle|rectangle,edge/noedge,left/right

\name{Tomás Capretto}{}

\position{Doctorate Candidate}
\address{IMASL/CONICET}

\email{\href{mailto:tomicapretto@gmail.com}{\nolinkurl{tomicapretto@gmail.com}}}
\homepage{tcapretto.netlify.com}
\github{tomicapretto}

% \gitlab{gitlab-id}
% \stackoverflow{SO-id}{SO-name}
% \skype{skype-id}
% \reddit{reddit-id}

\quote{Statistician interested in R, Python, R Shiny and anything that improves how we do data analysis}

\usepackage{booktabs}

% Templates for detailed entries
% Arguments: what when with where why
\usepackage{etoolbox}
\def\detaileditem#1#2#3#4#5{%
\cventry{#1}{#3}{#4}{#2}{\ifx#5\empty\else{\begin{cvitems}#5\end{cvitems}}\fi}\ifx#5\empty{\vspace{-4.0mm}}\else\fi}
\def\detailedsection#1{\begin{cventries}#1\end{cventries}}

% Templates for brief entries
% Arguments: what when with
\def\briefitem#1#2#3{\cvhonor{}{#1}{#3}{#2}}
\def\briefsection#1{\begin{cvhonors}#1\end{cvhonors}}

\providecommand{\tightlist}{%
	\setlength{\itemsep}{0pt}\setlength{\parskip}{0pt}}

%------------------------------------------------------------------------------



\begin{document}

% Print the header with above personal informations
% Give optional argument to change alignment(C: center, L: left, R: right)
\makecvheader

% Print the footer with 3 arguments(<left>, <center>, <right>)
% Leave any of these blank if they are not needed
% 2019-02-14 Chris Umphlett - add flexibility to the document name in footer, rather than have it be static Curriculum Vitae
\makecvfooter
  {diciembre 2020}
    {Tomás Capretto~~~·~~~Curriculum Vitae}
  {\thepage}


%-------------------------------------------------------------------------------
%	CV/RESUME CONTENT
%	Each section is imported separately, open each file in turn to modify content
%------------------------------------------------------------------------------



\hypertarget{education}{%
\section{Education}\label{education}}

\detailedsection{\detaileditem{Doctorate in Statistics}{2019-2024 (expected)}{Universidad Nacional de Rosario}{Rosario, Argentina}{\item{I work under the supervision of PhD. Osvaldo Martin on Bayesian statistics and exploratory data analysis.}\item{Studied density estimators and implemented a new alternative for the Python package ArviZ.This implementation was then translated to R and included in bayesplot package.}\item{Currently focused on Bambi, a Python library that makes Bayesian model building easier.}}\detaileditem{Bachelor in Statistics}{2014-2019}{Universidad Nacional de Rosario}{Rosario, Argentina}{\item{The career gave me a solid background in foundations of statistics as well as the ability to interpret results in an analytical manner.}}}

\hypertarget{research-experience}{%
\section{Research experience}\label{research-experience}}

\detailedsection{\detaileditem{Research assistant}{Apr 2018 - Mar 2019}{Universidad Nacional de Rosario}{Rosario, Argentina}{\item{\textbf{Project:} Statistical methods in official statistics.}\item{Developed my undergraduate thesis \textit{Estimation in complex survey samples in presence of nonresponse}.}}\detaileditem{Research assistant}{Apr 2016 - Mar 2017}{Universidad Nacional de Rosario}{Rosario, Argentina}{\item{\textbf{Project:} Sampling designs based on models for populations with spatial variability.}\item{As a complement of a course in sampling techniques, I dedicated most of the time to the study of new sampling methods that incorporate spatial variability.}}}

\hypertarget{work-experience}{%
\section{Work experience}\label{work-experience}}

\detailedsection{\detaileditem{Consultant}{Nov 2018 - Jan 2019}{AlixPartners}{Buenos Aires, Argentina}{\item{Data manipulation and analysis using R, Python, SQL and Tableau.}\item{Time series forecasting.}\item{Task automation.}}\detaileditem{Statistical assistant}{May 2018 - Oct 2018}{Instituto Nacional de Estadística y Censos}{Buenos Aires, Argentina}{\item{I worked on the study of non-response, its classification and treatment.}\item{Contributed to methodological manuals and internal reports.}\item{Evaluation of methods to impute missing data.}}\detaileditem{Statistical assistant}{Jul 2016 - Sep 2016}{Universidad Nacional de Rosario}{Rosario, Argentina}{\item{Survey sampling study to study characteristics of members from a professional college in Santa Fe province.}\item{Participated in the strategy to collect the data.}\item{Worked on the consistency of the data bases and the generation of internal reports.}\item{Worked on telephone surveys and cases follow-up.}}}

\pagebreak

\hypertarget{training-experience}{%
\section{Training experience}\label{training-experience}}

\detailedsection{\detaileditem{Online tutor}{Aug 2018 - May 2019}{Chegg Tutors}{Remote}{\item{Tutored Statistics, R and SAS Programming, and related topics in English.}\item{I worekd with both undergraduate and graduate students from a very wide range of disciplines.}}\detaileditem{Online tutor}{Nov 2015 - Nov 2018}{Latinhire}{Remote}{\item{Tutored Statistics both in English and Spanish.}\item{Worked with both high-school and undergraduate students.}}}

\hypertarget{scholarships-awards-and-distinctions}{%
\section{Scholarships, awards and distinctions}\label{scholarships-awards-and-distinctions}}

\detailedsection{\detaileditem{PhD scholarship in strategic topics}{2019 - 2024}{CONICET}{San Luis, Argentina}{\item{\textbf{Topic:} Exploratory analysis of Bayesian methods.}\item{Full-time position dedicated to study and research.}}\detaileditem{1st Prize in the XIII Young Biometrists Contest}{Oct 2017}{Grupo Argentino de Biometría}{Rosario, Argentina}{\item{I worked with David Presutti, Agronomic Engineering student.}\item{We used generalized mixed linear models to solve the proposed problem.}}\detaileditem{Fulbright scholar}{Jan 2017 - Feb 2017}{Comisión Fulbright Argentina}{Austin, United States}{\item{I made a stay in the  University of Texas at Austin thanks to the scholarship Friends of Fulbright 2017.}\item{The activities involved english training, cultural exchange, volunteering and I also attended classes in Statistics.}}}

\hypertarget{side-projects}{%
\section{Side projects}\label{side-projects}}

Out of curiosity or fun, I also dedicate part of my time to projects not directly
related to my doctorate.

\begin{itemize}
\tightlist
\item
  \href{https://github.com/tomicapretto/latex2r}{\texttt{latex2r}}:
  An R package to translate math formulas written in LaTeX to R code.
\item
  \href{https://github.com/tomicapretto/shinymath}{\texttt{shinymath}}: Math inputs in R Shiny apps.
  This package gives a new input that is based on Mathquill and returns R code.
\item
  \href{https://github.com/tomicapretto/formulae}{\texttt{formulae}}: A WIP library that
  enables using the popular R formulas in Python. While there is \href{https://github.com/pydata/patsy}{\texttt{patsy}}
  out there, this implementation also handles group specific effects
  (a.k.a. random effects), among other features.
\end{itemize}

\end{document}
